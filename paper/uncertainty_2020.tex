\documentclass{article}

% if you need to pass options to natbib, use, e.g.:
%     \PassOptionsToPackage{numbers, compress}{natbib}
% before loading neurips_2020

% ready for submission
% \usepackage{neurips_2020}

% to compile a preprint version, e.g., for submission to arXiv, add add the
% [preprint] option:
%     \usepackage[preprint]{neurips_2020}

% to compile a camera-ready version, add the [final] option, e.g.:
%     \usepackage[final]{neurips_2020}

% to avoid loading the natbib package, add option nonatbib:
\usepackage[nonatbib]{neurips_2020}

\usepackage[utf8]{inputenc} % allow utf-8 input
\usepackage[T1]{fontenc}    % use 8-bit T1 fonts
\usepackage{hyperref}       % hyperlinks
\usepackage{url}            % simple URL typesetting
\usepackage{booktabs}       % professional-quality tables
\usepackage{amsfonts}       % blackboard math symbols
\usepackage{nicefrac}       % compact symbols for 1/2, etc.
\usepackage{microtype}      % microtypography


% additional packages
\usepackage{amsmath}
\usepackage{amssymb}
\usepackage{color}
\usepackage{amsthm}
\usepackage{subcaption}
\usepackage[numbers]{natbib}
\usepackage{graphicx}



% tikz pictures
%\usepackage[usenames,dvipsnames,pdftex,table]{xcolor}
\usepackage{tikz, ifthen}
\usetikzlibrary{angles, arrows, arrows.meta,calc,chains,shapes,fit,positioning,decorations.pathmorphing,quotes}
\usetikzlibrary{arrows.meta,calc,chains,shapes,fit,positioning,decorations.pathmorphing}
\def\layersep{3.2cm}
\usetikzlibrary{patterns, shapes, calc}

\usepackage{color}
\usepackage{colortbl}
%\usepgfplotslibrary{colorbrewer}

\usetikzlibrary{arrows,positioning} 
\tikzset{
	%Define standard arrow tip
	>=stealth',
	%Define style for boxes
	punkt/.style={
		rectangle,
		rounded corners,
		draw=black, very thick,
		text width=6.5em,
		minimum height=2em,
		text centered},
	% Define arrow style
	pil/.style={
		->,
		thick,
		shorten <=2pt,
		shorten >=2pt,}
	% arrows 
	state/.style={circle,draw,minimum size=6ex},
	arrow/.style={-latex, shorten >=1ex, shorten <=1ex}
}

\usepackage{pgf}


\usepackage{subcaption,graphicx}
\newcommand{\rulesep}{\unskip\ \vrule\ }

\newcommand\pgfmathsinandcos[3]{% 
	\pgfmathsetmacro#1{sin(#3)} 
	\pgfmathsetmacro#2{cos(#3)}}


\usepackage{booktabs}
\newcommand{\ra}[1]{\renewcommand{\arraystretch}{#1}}

% additional commands
\DeclareMathOperator*{\argmin}{\arg\!\min} 
\DeclareMathOperator*{\argmax}{\arg\!\max} 


\newcommand{\todo}[1]{\textcolor{red}{(TODO: #1)}}
\newcommand{\sg}[1]{\textcolor{blue}{(SG: #1)}}


















\title{Robustness of Predictive Uncertainty Estimation Techniques}

% The \author macro works with any number of authors. There are two commands
% used to separate the names and addresses of multiple authors: \And and \AND.
%
% Using \And between authors leaves it to LaTeX to determine where to break the
% lines. Using \AND forces a line break at that point. So, if LaTeX puts 3 of 4
% authors names on the first line, and the last on the second line, try using
% \AND instead of \And before the third author name.

\author{%
  Anna-Kathrin Kopetzki \\
  Department of Informatics \\
  Technical University of Munich \\
  Munich, Germany \\
  \texttt{anna.k.kopetzki@tum.de} \\
  % examples of more authors
  \And
   Sandhya Giri \\
   Department of Informatics \\
   Technical University of Munich \\
   Munich, Germany \\
   \texttt{sandhya.giri@tum.de} \\
   % examples of more authors
   \And
    Stephan G\"unnemann \\
    Department of Informatics \\
    Technical University of Munich \\
    \texttt{guennemann@in.tum.de} \\
}

\begin{document}

\maketitle



\section{Experiments}



\begin{table}[ht]
	\centering
	\ra{1.3}
	\caption{Randomized smoothing for robustness verification of prior networks with $\sigma=0.2$ on $10^3$ samples. For verification radius~$r$ and accuracy (acc. in $[\%]$) are computed with respect to the following measures: confidence ($m_{\mathrm{conf}}$), differential entropy ($m_{\mathrm{diffE}}$.) and distributional uncertainty ($m_{\mathrm{distU}}$). We compare a normally trained model (Normal), models trained on adversarial ($A$) computed with respect to different uncertainty measures (i=train on in-to-out adversarial examples, o=trained on out-to-in adversarials) and a robustly constructed prior-network (RPN).}
	\begin{tiny}
		\begin{tabular}{@{}rrrrrrrcrrrrcrrrr@{}}
			\toprule
			& \multicolumn{6}{c}{in-distribution} &  & \multicolumn{4}{c}{out-distribution I} &   & \multicolumn{4}{c}{out-distribution II} \\
			%\cmidrule{2-7} \cmidrule{9-12} \cmidrule{14-17}
			& \multicolumn{2}{c}{$m_{\mathrm{conf}}$}  & \multicolumn{2}{c}{$m_{\mathrm{diffE}}$} &  \multicolumn{2}{c}{$m_{\mathrm{distU}}$} &   
			& \multicolumn{2}{c}{$m_{\mathrm{diffE}}$} & \multicolumn{2}{c}{$m_{\mathrm{distU}}$} & 
			& \multicolumn{2}{c}{$m_{\mathrm{diffE}}$} & \multicolumn{2}{c}{$m_{\mathrm{distU}}$} \\
			\cmidrule{2-3}  \cmidrule{4-5} \cmidrule{6-7} \cmidrule{9-10}  \cmidrule{11-12} \cmidrule{14-15} \cmidrule{16-17}
			model & r & acc. & r & acc. & r & acc. & & r & acc. & r & acc. & & r & acc. & r & acc. \\
			\midrule
			& \multicolumn{6}{c}{MNIST} & & \multicolumn{4}{c}{OMNIGLOT} & & \multicolumn{4}{c}{CIFAR10} \\
			Normal                        & 0.452 & 98 & 0.471 & 100 & 0.438 & 99 & & 0.472 & 100 & 0.472 & 100 & & 0.411 & 95 & 0.343 & 93 \\
			$\mathrm{A}_{\mathrm{conf}}$  & 0.468 & 99 & 0.472 & 100 & 0.470 & 100 & & 0.470 & 100 & 0.471 & 100 & & 0.326 & 41 & 0.278 & 61 \\
			$\mathrm{A}_{\mathrm{diffE}}$ & 0.468 & 99 & 0.472 & 100 & 0.468 & 100 & & 0.472 & 100 & 0.472 & 100 & & 0.362 & 49 & 0.320 & 64 \\
			$\mathrm{A}_{\mathrm{diffE_b}}$ & 0.465 & 99 & 0.458 & 99 & 0.457 & 98 & & 0.472 & 100 & 0.472 & 100 & & 0.469 & 100 & 0.465 & 100 \\
			$\mathrm{A}_{\mathrm{distU}}$ & 0.469 & 99 & 0.472 & 100 & 0.472 & 100 & & 0.472 & 100 & 0.472 & 100 & & 0.370 & 59 & 0.384 & 28 \\
			$\mathrm{A}_{\mathrm{distU_b}}$ & 0.468 & 99 & 0.472 & 100 & 0.472 & 100 & & 0.472 & 100 & 0.472 & 100 & & 0.434 & 92 & 0.428 & 93 \\
			RPN                           & 0.466 & 99 & 0.472 & 100 & 0.465 & 99 & & 0.472 & 100 & 0.471 & 100 & & 0.358 & 68 & 0.301 & 82 \\
			\midrule
			& \multicolumn{6}{c}{MNIST}   & & \multicolumn{4}{c}{CIFAR10} & & \multicolumn{4}{c}{OMNIGLOT} \\
			Normal                         & .. & .. & .. & .. & .. & .. & & .. & .. & .. & .. & & .. & .. & .. & .. \\
			$\mathrm{A}_{\mathrm{conf}}$  & .. & .. & .. & .. & .. & .. & & .. & .. & .. & .. & & .. & .. & .. & .. \\
			$\mathrm{A}_{\mathrm{diffE}}$ & .. & .. & .. & .. & .. & .. & & .. & .. & .. & .. & & .. & .. & .. & .. \\
			$\mathrm{A}_{\mathrm{distU}}$ & .. & .. & .. & .. & .. & .. & & .. & .. & .. & .. & & .. & .. & .. & .. \\
			RPN                           & .. & .. & .. & .. & .. & .. & & .. & .. & .. & .. & & .. & .. & .. & .. \\
			\midrule
			& \multicolumn{6}{c}{FashionMNIST} & & \multicolumn{4}{c}{SVHN} & & \multicolumn{4}{c}{CIFAR10} \\
			Normal                         & .. & .. & .. & .. & .. & .. & & .. & .. & .. & .. & & .. & .. & .. & .. \\
			$\mathrm{A}_{\mathrm{conf}}$  & .. & .. & .. & .. & .. & .. & & .. & .. & .. & .. & & .. & .. & .. & .. \\
			$\mathrm{A}_{\mathrm{diffE}}$ & .. & .. & .. & .. & .. & .. & & .. & .. & .. & .. & & .. & .. & .. & .. \\
			$\mathrm{A}_{\mathrm{distU}}$ & .. & .. & .. & .. & .. & .. & & .. & .. & .. & .. & & .. & .. & .. & .. \\
			RPN                           & .. & .. & .. & .. & .. & .. & & .. & .. & .. & .. & & .. & .. & .. & .. \\
			\midrule
			& \multicolumn{6}{c}{CIFAR10} & & \multicolumn{4}{c}{SVHN} & & \multicolumn{4}{c}{TIM} \\
			Normal                         & .. & .. & .. & .. & .. & .. & & .. & .. & .. & .. & & .. & .. & .. & .. \\
			$\mathrm{A}_{\mathrm{conf}}$  & .. & .. & .. & .. & .. & .. & & .. & .. & .. & .. & & .. & .. & .. & .. \\
			$\mathrm{A}_{\mathrm{diffE}}$ & .. & .. & .. & .. & .. & .. & & .. & .. & .. & .. & & .. & .. & .. & .. \\
			$\mathrm{A}_{\mathrm{distU}}$ & .. & .. & .. & .. & .. & .. & & .. & .. & .. & .. & & .. & .. & .. & .. \\
			RPN                           & .. & .. & .. & .. & .. & .. & & .. & .. & .. & .. & & .. & .. & .. & .. \\
			\midrule
			& \multicolumn{6}{c}{CIFAR10} & & \multicolumn{4}{c}{LSUN} & & \multicolumn{4}{c}{TIM} \\
			Normal                         & .. & .. & .. & .. & .. & .. & & .. & .. & .. & .. & & .. & .. & .. & .. \\
			$\mathrm{A}_{\mathrm{conf}}$  & .. & .. & .. & .. & .. & .. & & .. & .. & .. & .. & & .. & .. & .. & .. \\
			$\mathrm{A}_{\mathrm{diffE}}$ & .. & .. & .. & .. & .. & .. & & .. & .. & .. & .. & & .. & .. & .. & .. \\
			$\mathrm{A}_{\mathrm{distU}}$ & .. & .. & .. & .. & .. & .. & & .. & .. & .. & .. & & .. & .. & .. & .. \\
			RPN                           & .. & .. & .. & .. & .. & .. & & .. & .. & .. & .. & & .. & .. & .. & .. \\
			\bottomrule
		\end{tabular}
	\end{tiny}
	\label{tab:res_smoothing}
\end{table}













%Papers may only be up to eight pages long, including figures. Additional pages \emph{containing only a section on the broader impact, acknowledgments and/or cited references} are allowed. Papers that exceed eight pages of content will not be reviewed, or in any other way considered for presentation at the conference.









\end{document}
